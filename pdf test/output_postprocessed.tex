{}

\begin{longtable}[]{@{}l@{}}
\toprule\noalign{}
\endhead
\bottomrule\noalign{}
\endlastfoot
 \\
 \\
 \\
 \\
 \\
\end{longtable}

\hfill\break

Thời gian làm bài: 240 phút

\subsubsection{Ngày thi: 11-12-2024}\label{nguxe0y-thi-11-12-2024}

\hfill\break

\subsubsection{TỔNG QUAN ĐỀ THI}\label{tux1ed5ng-quan-ux111ux1ec1-thi}

Bài 1. Biến đổi xâu --- TRANSF (100 điểm){ }2

Bài 2. Ngôi nhà mới --- NEWHOME (100 điểm){ }4

Bài 3. Trò chơi trên vòng tròn --- CIRCLE (100 điểm){ }6

Bài 4. Chia xâu --- BSTRING (100 điểm){ }7

Lưu ý:

\begin{itemize}
\item
  Điểm của mỗi bài là tổng điểm của các subtask. Điểm của subtask với
  mỗi lần nộp là điểm nhỏ nhất đạt được của các test trong subtask đó;
\item
  Thí sinh không được phép sử dụng các định hướng biên dịch chương trình
  có các từ khoá sau: {pragma, optimize, target, O3, Ofast,
  unroll-loops, avx, avx2, fma, omit-frame-pointer}.
\end{itemize}

\section{Bài 1. Biến đổi xâu ---
TRANSF}\label{buxe0i-1.-biux1ebfn-ux111ux1ed5i-xuxe2u-transf}

\hfill\break

Alice có hai xâu {S }và {T }, mỗi xâu gồm các ký tự trong tập từ 'a' đến
'z'. Xâu {S }và xâu {T }có độ dài lần lượt là {ℓ}{S}{ }và {ℓ}{T}{ }.
Alice muốn biến đổi xâu {S }và {T }để chúng giống nhau bằng việc chèn
thêm các xâu khác vào các vị trí bất kỳ. Có {K }xâu cho trước (gọi là
\it{xâu thêm}) được đánh số từ {1 }đến {K}, mỗi xâu có thể được sử
dụng không giới hạn số lần để Alice chèn vào xâu {S }và {T }.

Việc chèn xâu thêm vào một xâu {Z }(với {Z }là xâu {S }hoặc xâu {T })
cần được thực hiện với các lưu ý sau:

\hfill\break

\begin{itemize}
\item
  {Xâu }Z {có độ dài }ℓ {gồm các ký tự }Z{1}Z{2}{ }. . . Z{ℓ}{. Mỗi vị
  trí chèn được đánh số từ }{0 }{đến }ℓ{. Vị trí }{0 }{là vị trí ngay
  bên trái ký tự }Z{1}{. Vị trí }i {(}{1 }{≤ }i {≤ }ℓ{) là vị trí ngay
  bên phải ký tự }Z{i}{.}
\item
  Một xâu được chèn vào {Z }có thể thực hiện tại bất kỳ vị trí nào trong
  xâu, bao gồm trước ký tự đầu tiên và sau ký tự cuối cùng.
\item
  {Ta có thể chèn nhiều xâu vào cùng một vị trí. Các xâu thêm được chèn
  lần lượt theo một thứ tự xác định. Ví dụ, với }Z {= }Z{1}Z{2}Z{3}{, ba
  xâu thêm }A {= }A{1}A{2}A{3}{, }B {= }B{1}B{2}{ }{và }C {=
  }C{1}C{2}C{3}C{4}{ }{được chèn lần lượt vào vị trí 1, ta thu được xâu
  }Z{1}C{1}C{2}C{3}C{4}B{1}B{2}A{1}A{2}A{3}Z{2}Z{3}{.}
\item
  Vị trí chèn các xâu là cố định kể cả sau khi đã chèn một số xâu. Ví
  dụ, với {Z }{=}"aabc", nếu ta chèn "dd" vào vị trí {0}; "e" và "gg"
  vào vị trí {1}; "f" vào vị trí {4 }thì ta có xâu kết quả là {Z
  }{=}"ddaggeabcf".

  \hfill\break

  \textbf{Yêu cầu:} Hãy giúp Alice chèn các xâu thêm sao cho xâu {S
  }giống với xâu {T }và tổng số xâu chèn thêm vào là ít nhất.

  \subsection{Dữ liệu}\label{dux1eef-liux1ec7u}

  \hfill\break
\item
  {Dòng đầu tiên chứa xâu }S {(}{0 }{≤ }ℓ{S} {≤ }{300}{).}
\item
  {Dòng thứ hai chứa xâu }T {(}{0 }{≤ }ℓ{T} {≤ }{300}{).}
\item
  Dòng thứ ba chứa số nguyên {K }({1 }{≤ }{K }{≤ }{30}), là số lượng xâu
  có thể sử dụng để chèn.
\item
  {Mỗi dòng trong số }K {dòng tiếp theo chứa một xâu }X{i} {(}{1 }{≤ }i
  {≤ }K{), biểu diễn xâu thêm thứ }i{. Tổng độ dài của }X{i} {không vượt
  quá 30.}

  \hfill\break

  \subsection{Kết quả}\label{kux1ebft-quux1ea3}

  \hfill\break
\item
  Nếu không thể biến đổi để {S }giống {T }, ghi duy nhất một số {−}{1}.
  Ngược lại, ghi ra số lượng xâu cần phải chèn.
\item
  {Dòng thứ }i {trong số }ℓ{S} {+ 1 }{dòng tiếp theo ghi ra một số
  nguyên }P{i} {là số lượng xâu được chèn ở vị trí }i {trong xâu }S{.
  Nếu }P{i} > {0 }{thì }P{i} {số tiếp theo ghi ra các thứ
  tự các xâu được chèn lần lượt ở vị trí }i{.}
\item
  {Dòng thứ }j {trong số }ℓ{T} {+ 1 }{dòng tiếp theo ghi ra }Q{j} {là số
  lượng xâu được chèn ở vị trí }j {trong xâu}

  T {. Nếu }Q{j} > {0 }{thì }Q{j} {số tiếp theo ghi ra các
  thứ tự các xâu được chèn lần lượt ở vị trí }i{.}

  \subsection{Ví dụ}\label{vuxed-dux1ee5}

  \hfill\break

  \begin{longtable}[]{@{}
    >{\raggedright\arraybackslash}p{(\linewidth - 2\tabcolsep) * \real{0.5000}}
    >{\raggedright\arraybackslash}p{(\linewidth - 2\tabcolsep) * \real{0.5000}}@{}}
  \toprule\noalign{}
  \endhead
  \bottomrule\noalign{}
  \endlastfoot
  stdin & stdout \\
  aaa

  b 1

  aab & -1 \\
  aad

  bb 3

  bbd ad d & 5

  0

  2 1 3

  0

  0

  1 2

  0

  2 2 3 \\
  \end{longtable}

  \subsection{Giải thích}\label{giux1ea3i-thuxedch}

  Trong ví dụ thứ hai:

  \hfill\break
\item
  Với xâu {S}, cần 2 phép chèn tại vị trí 1. Xâu "bbd" và "d" được chèn
  tại vị trí 1.
\item
  Với xâu {T }, chèn "ad" tại vị trí 0 và chèn "ad" và "d" tại vị trí 2.
\item
  Xâu kết quả của cả {S }và {T }đều là "adbbdad".

  \hfill\break

  \subsection{Cách tính điểm}\label{cuxe1ch-tuxednh-ux111iux1ec3m}

  \hfill\break
\item
  {Subtask 1 (30\% số điểm): Độ dài tất cả các xâu }X{i} {(}{1 }{≤ }i {≤
  }K{) bằng }{1}{.}
\item
  {Subtask 2 (30\% số điểm): Các xâu }S{, }T {và }X{i} {(}{1 }{≤ }i {≤
  }K{) chỉ chứa ký tự 'a' và/hoặc 'b'. Độ dài tất cả các xâu }X{i}
  {không quá }{2}{.}
\item
  Subtask 3 (40\% số điểm): Không có ràng buộc nào thêm.

  \section{Bài 2. Ngôi nhà mới ---
  NEWHOME}\label{buxe0i-2.-nguxf4i-nhuxe0-mux1edbi-newhome}

  \hfill\break

  Ở ngôi làng nọ có {N }ngôi nhà, đánh số từ 1 đến {N }. Có {N }{− }{1
  }con đường, mỗi con đường nối một cặp ngôi nhà và cho phép đi lại theo
  cả hai chiều. Hệ thống đường bảo đảm đi lại giữa mọi ngôi nhà.

  {Có }N {công dân sống trong làng, mỗi ngôi nhà có đúng một công dân.
  Công dân ở ngôi nhà thứ }i {thích màu }C{(}i{)}{, vì vậy người đó đã
  sơn màu }C{(}i{) }{cho ngôi nhà thứ }i{.}

  {Trưởng làng muốn đổi chỗ ở cho người dân của mình. Ông chọn ra ngẫu
  nhiên một hoán vị }P {= }P{1}, P{2}, . . . , P{N} {thoả mãn }C{(}i{) =
  }C{(}P{i}{) }{với mọi }i {= 1}, {2}, {3}, . . . , N {(đây là yêu cầu
  về màu sắc). Các hoán vị thoả mãn đều có xác suất được chọn như nhau,
  và ông chỉ chọn trong số các hoán vị thoả mãn yêu cầu về màu sắc mà
  thôi. Tiếp đến, ông ra lệnh cho người ở ngôi nhà thứ }i {chuyển nhà
  tới ngôi nhà thứ }P{i}{, khi đó người này vẫn được ở trong ngôi nhà có
  màu yêu thích của mình; độ dài quãng đường mà người này phải đi là số
  cạnh trên đường đi đơn giữa }i {và }P{i}{. Chi phí để đổi nhà theo
  hoán vị }P {được tính bằng tổng độ dài đường đi của tất cả công dân.
  Hãy tính kỳ vọng chi phí đổi nhà.}

  {Nhắc lại: }{Kỳ vọng của một biến ngẫu nhiên }X {rời rạc được tính
  bởi: }{Σ }X{(}A{) }{× }p{(}A{) }{với }{P }{là tập}

  A{∈P}

  {tất cả các biến cố có thể xảy ra, }p{(}A{) }{là xác suất xảy ra biến
  cố }A{, và }X{(}A{) }{là giá trị của biến ngẫu nhiên }X {khi xảy ra
  biến cố }A{. Cụ thể trong trường hợp của bài này: }{P }{là tập tất cả
  các hoán vị thoả mãn yêu cầu về màu sắc; }A {là một hoán vị thuộc
  }{P}{; }p{(}A{) }{là xác suất mà trưởng làng chọn hoán vị }A{; và
  }X{(}A{) }{là chi phí để đổi nhà theo hoán vị }A{.}

  \subsection{Dữ liệu}\label{dux1eef-liux1ec7u-1}

  \hfill\break
\item
  Dòng đầu tiên chứa số nguyên dương {N }.
\item
  {Dòng thứ hai chứa }N {số nguyên không âm }C{(1)}, C{(2)}, C{(3)}, . .
  . , C{(}N {)}{.}
\item
  Mỗi dòng trong số {N }{− }{1 }dòng tiếp theo chứa hai số nguyên dương
  {u, v }cho biết có một con đường nối giữa ngôi nhà thứ {u }và ngôi nhà
  thứ {v}.

  \hfill\break

  \subsection{Kết quả}\label{kux1ebft-quux1ea3-1}

  Q

  Kết quả là một số hữu tỷ, được biểu diễn dưới dạng phân số tối giản
  {P}{ }, hãy in ra {P }{×}

  Q{1000000005}{\%1000000007}{.}

  \subsection{Ví dụ}\label{vuxed-dux1ee5-1}

  \hfill\break

  \begin{longtable}[]{@{}
    >{\raggedright\arraybackslash}p{(\linewidth - 2\tabcolsep) * \real{0.5000}}
    >{\raggedright\arraybackslash}p{(\linewidth - 2\tabcolsep) * \real{0.5000}}@{}}
  \toprule\noalign{}
  \endhead
  \bottomrule\noalign{}
  \endlastfoot
  stdin & stdout \\
  6

  1 4 2 4 1 4

  1 2

  1 3

  2 4

  2 5

  3 6 & 333333343 \\
  \end{longtable}

  \subsection{Giải thích}\label{giux1ea3i-thuxedch-1}

  Có 12 hoán vị thoả mãn, với tổng chi phí là 88.

  {}

  \begin{longtable}[]{@{}l@{}}
  \toprule\noalign{}
  \endhead
  \bottomrule\noalign{}
  \endlastfoot
   \\
  \end{longtable}

  \hfill\break

  \subsection{Cách tính điểm}\label{cuxe1ch-tuxednh-ux111iux1ec3m-1}

  \hfill\break
\item
  {Trong tất cả các test: }{N, C}({i}) {≤ }3 {× }10{5}{;}
\item
  Subtask 1 (12 điểm): {N }{≤ }{10}.
\item
  Subtask 2 (16 điểm): Mỗi ngôi nhà chỉ có đúng một ngôi nhà khác có
  cùng màu với nó.
\item
  Subtask 3 (20 điểm): {N }{≤ }{1000}.
\item
  {Subtask 4 (24 điểm): }C{(1) = }C{(2) = }. . . {= }C{(}n{)}{.}
\item
  Subtask 5 (12 điểm): Không có ràng buộc nào thêm.
\end{itemize}

\section{Bài 3. Trò chơi trên vòng tròn ---
CIRCLE}\label{buxe0i-3.-truxf2-chux1a1i-truxean-vuxf2ng-truxf2n-circle}

\hfill\break

Cho {M }tập số nguyên rời rạc, trong đó mỗi số nguyên từ {1 }đến {N
}thuộc duy nhất một trong {M }tập. Các tập {1}{, }{2}{, }{3}{, , ..., M
}được sắp xếp liên tiếp trên một vòng tròn theo chiều kim đồng hồ (tập
{M }kề với tập {1}).

Thực hiện quy trình sau đây lặp đi lặp lại:

\hfill\break

\begin{enumerate}
\item
  Tại mỗi bước, chọn số nhỏ nhất trong mỗi tập và loại bỏ số đó khỏi
  tập.
\item
  Số nhỏ nhất được loại bỏ khỏi tập {i }sẽ được thêm vào tập {i }{+ 1
  }(với tập {M }thì thêm vào tập {1}).
\item
  Nếu tập {i }rỗng thì không có số nào bị loại bỏ khỏi tập {i }và thêm
  vào tập kề {i }{+ 1 }(với số bị loại bỏ khỏi tập {M }thì thêm vào tập
  {1}).

  \hfill\break

  Có {Q }truy vấn, mỗi truy vấn yêu cầu xác định số nhỏ nhất bị loại bỏ
  tại lần chuyển thứ {T }của tập {1}, nếu tại lần chuyển thứ {T }mà tập
  {1 }không có số nào thì đưa ra {−}{1}.

  \subsection{Dữ liệu}\label{dux1eef-liux1ec7u-2}

  \hfill\break

  \begin{itemize}
  \item
    {Dòng đầu tiên chứa ba số nguyên dương }N, M, Q {(}N, M, Q {≤ }{300
    000)}{;}
  \item
    {Dòng thứ hai chứa }N {số nguyên dương }G{1}, G{2}, ..., G{n} {(1
    }{≤ }G{i} {≤ }M {)}{, với }G{i} {là tập mà số }i {ban đầu thuộc;}
  \item
    {Dòng tiếp theo chứa }Q {số nguyên }T{1}, T{2}, ..., T{Q} {(1 }{≤
    }T{j} {≤ }{10}{9}{) }{tương ứng với các truy vấn.}

    \hfill\break

    \subsection{Kết quả}\label{kux1ebft-quux1ea3-2}

    Ghi ra kết quả trên một dòng gồm {Q }số tương ứng với kết quả của
    mỗi truy vấn.

    \subsection{Cách tính điểm}\label{cuxe1ch-tuxednh-ux111iux1ec3m-2}

    \hfill\break
  \item
    {Subtask 1 (9 điểm): }N, M, Q, T{j} {≤ }{1 000}{;}
  \item
    Subtask 2 (23 điểm): {M }{≤ }{1 000};
  \item
    Subtask 3 (35 điểm): {Q }{= 1};
  \item
    Subtask 4 (33 điểm): không có ràng buộc gì thêm.

    \hfill\break

    \subsection{Ví dụ}\label{vuxed-dux1ee5-2}

    \hfill\break

    \begin{longtable}[]{@{}
      >{\raggedright\arraybackslash}p{(\linewidth - 12\tabcolsep) * \real{0.1429}}
      >{\raggedright\arraybackslash}p{(\linewidth - 12\tabcolsep) * \real{0.1429}}
      >{\raggedright\arraybackslash}p{(\linewidth - 12\tabcolsep) * \real{0.1429}}
      >{\raggedright\arraybackslash}p{(\linewidth - 12\tabcolsep) * \real{0.1429}}
      >{\raggedright\arraybackslash}p{(\linewidth - 12\tabcolsep) * \real{0.1429}}
      >{\raggedright\arraybackslash}p{(\linewidth - 12\tabcolsep) * \real{0.1429}}
      >{\raggedright\arraybackslash}p{(\linewidth - 12\tabcolsep) * \real{0.1429}}@{}}
    \toprule\noalign{}
    \endhead
    \bottomrule\noalign{}
    \endlastfoot
    \multicolumn{5}{@{}>{\raggedright\arraybackslash}p{(\linewidth - 12\tabcolsep) * \real{0.7143} + 8\tabcolsep}}{%
    stdin} &
    \multicolumn{2}{>{\raggedright\arraybackslash}p{(\linewidth - 12\tabcolsep) * \real{0.2857} + 2\tabcolsep}@{}}{%
    stdout} \\
    5 & 3 & 2 & \begin{minipage}[t]{\linewidth}\raggedright
    \hfill\break
    \strut
    \end{minipage} & \begin{minipage}[t]{\linewidth}\raggedright
    \hfill\break
    \strut
    \end{minipage} & 1 & 2 \\
    1 & 1 & 2 & 2 & 3 & \begin{minipage}[t]{\linewidth}\raggedright
    \hfill\break
    \strut
    \end{minipage} & \begin{minipage}[t]{\linewidth}\raggedright
    \hfill\break
    \strut
    \end{minipage} \\
    1 & 2 & \begin{minipage}[t]{\linewidth}\raggedright
    \hfill\break
    \strut
    \end{minipage} & \begin{minipage}[t]{\linewidth}\raggedright
    \hfill\break
    \strut
    \end{minipage} & \begin{minipage}[t]{\linewidth}\raggedright
    \hfill\break
    \strut
    \end{minipage} & \begin{minipage}[t]{\linewidth}\raggedright
    \hfill\break
    \strut
    \end{minipage} & \begin{minipage}[t]{\linewidth}\raggedright
    \hfill\break
    \strut
    \end{minipage} \\
    \end{longtable}

    \section{Bài 4. Chia xâu ---
    BSTRING}\label{buxe0i-4.-chia-xuxe2u-bstring}

    \hfill\break

    Trong khoa học máy tính, biểu diễn và xử lý các số nhị phân là một
    phần quan trọng của nhiều bài toán thực tế. Một số nhị phân có càng
    nhiều bit 1 thì mức độ phức tạp khi lưu trữ hoặc xử lý càng lớn. Vì
    thế, việc tối thiểu hóa số lượng bit 1 trong các phép tính số nhị
    phân là một bài toán có ý nghĩa thiết thực.

    Với một xâu nhị phân, bạn cần chia xâu này thành các đoạn liên tiếp.
    Mỗi đoạn sẽ được chuyển đổi thành một số nguyên dưới dạng nhị phân.
    Sau đó, tính tổng các đoạn này.

    \paragraph{\texorpdfstring{Yêu cầu: {Tìm cách chia xâu sao cho tổng
    các đoạn đó, khi biểu diễn ở dạng nhị phân, chứa số lượng bit 1 ít
    nhất.}}{Yêu cầu: Tìm cách chia xâu sao cho tổng các đoạn đó, khi biểu diễn ở dạng nhị phân, chứa số lượng bit 1 ít nhất.}}\label{yuxeau-cux1ea7u-tuxecm-cuxe1ch-chia-xuxe2u-sao-cho-tux1ed5ng-cuxe1c-ux111oux1ea1n-ux111uxf3-khi-biux1ec3u-diux1ec5n-ux1edf-dux1ea1ng-nhux1ecb-phuxe2n-chux1ee9a-sux1ed1-lux1b0ux1ee3ng-bit-1-uxedt-nhux1ea5t.}

    \subsection{Dữ liệu}\label{dux1eef-liux1ec7u-3}

    \hfill\break
  \item
    Dòng đầu tiên chứa số nguyên dương {Q }({Q }{≤ }{100}) là số truy
    vấn.
  \item
    Mỗi dòng trong số {Q }dòng tiếp theo chứa xâu nhị phân trong truy
    vấn thứ {i}.

    \hfill\break

    \subsection{Kết quả}\label{kux1ebft-quux1ea3-3}

    Ghi ra {Q }dòng, dòng thứ {i }đưa ra tương ứng câu trả lời cho xâu ở
    truy vấn thứ {i}. Với mỗi xâu, thực hiện chèn dấu cộng vào các vị
    trí các điểm ngắt đoạn xâu. Không đưa ra dấu cách hoặc ký tự bất kỳ
    khác.

    \subsection{Ví dụ}\label{vuxed-dux1ee5-3}

    \hfill\break

    \begin{longtable}[]{@{}
      >{\raggedright\arraybackslash}p{(\linewidth - 2\tabcolsep) * \real{0.5000}}
      >{\raggedright\arraybackslash}p{(\linewidth - 2\tabcolsep) * \real{0.5000}}@{}}
    \toprule\noalign{}
    \endhead
    \bottomrule\noalign{}
    \endlastfoot
    stdin & stdout \\
    2 & 11+0+1101 \\
    1101101 & 10+1+0+1 \\
    10101 & \begin{minipage}[t]{\linewidth}\raggedright
    \hfill\break
    \strut
    \end{minipage} \\
    \end{longtable}

    \subsection{Cách tính điểm}\label{cuxe1ch-tuxednh-ux111iux1ec3m-3}

    Với mỗi xâu truy vấn, gọi {N }là độ dài xâu, {K }là số bit 1 trong
    một xâu truy vấn đó.

    \hfill\break
  \item
    Subtask 1 (15 điểm): {N }{≤ }{16}.
  \item
    {Subtask 2 (30 điểm): }N {≤ }{10}{5}, K {≤ }{10}{3}{.}
  \item
    Subtask 3 (55 điểm): {N }{≤ }{10}{5}.
  \end{itemize}
\end{enumerate}

\hfill\break

out

Với mỗi subtask, gồm nhiều test. Với mỗi test, gọi {out }là tổng số bit
1 của tất cả các truy vấn tính được theo cách phân tích của chương trình
thí sinh. {ans }là tổng số bit 1 của tất cả các truy vấn tính được theo
cách phân tích của ban giám khảo. Điểm của thí sinh nhận được cho test
đó được tính bằng {( }{ans}{ }{)}{2}.

Điểm của mỗi subtask được tính bằng điểm thấp nhất của một test trong
subtask đó.
