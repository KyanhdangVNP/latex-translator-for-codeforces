% Options for packages loaded elsewhere
\PassOptionsToPackage{unicode}{hyperref}
\PassOptionsToPackage{hyphens}{url}
\documentclass[
]{article}
\usepackage{xcolor}
\usepackage{amsmath,amssymb}
\setcounter{secnumdepth}{-\maxdimen} % remove section numbering
\usepackage{iftex}
\ifPDFTeX
  \usepackage[T1]{fontenc}
  \usepackage[utf8]{inputenc}
  \usepackage{textcomp} % provide euro and other symbols
\else % if luatex or xetex
  \usepackage{unicode-math} % this also loads fontspec
  \defaultfontfeatures{Scale=MatchLowercase}
  \defaultfontfeatures[\rmfamily]{Ligatures=TeX,Scale=1}
\fi
\usepackage{lmodern}
\ifPDFTeX\else
  % xetex/luatex font selection
\fi
% Use upquote if available, for straight quotes in verbatim environments
\IfFileExists{upquote.sty}{\usepackage{upquote}}{}
\IfFileExists{microtype.sty}{% use microtype if available
  \usepackage[]{microtype}
  \UseMicrotypeSet[protrusion]{basicmath} % disable protrusion for tt fonts
}{}
\makeatletter
\@ifundefined{KOMAClassName}{% if non-KOMA class
  \IfFileExists{parskip.sty}{%
    \usepackage{parskip}
  }{% else
    \setlength{\parindent}{0pt}
    \setlength{\parskip}{6pt plus 2pt minus 1pt}}
}{% if KOMA class
  \KOMAoptions{parskip=half}}
\makeatother
\usepackage{longtable,booktabs,array}
\usepackage{calc} % for calculating minipage widths
% Correct order of tables after \paragraph or \subparagraph
\usepackage{etoolbox}
\makeatletter
\patchcmd\longtable{\par}{\if@noskipsec\mbox{}\fi\par}{}{}
\makeatother
% Allow footnotes in longtable head/foot
\IfFileExists{footnotehyper.sty}{\usepackage{footnotehyper}}{\usepackage{footnote}}
\makesavenoteenv{longtable}
\usepackage{graphicx}
\makeatletter
\newsavebox\pandoc@box
\newcommand*\pandocbounded[1]{% scales image to fit in text height/width
  \sbox\pandoc@box{#1}%
  \Gscale@div\@tempa{\textheight}{\dimexpr\ht\pandoc@box+\dp\pandoc@box\relax}%
  \Gscale@div\@tempb{\linewidth}{\wd\pandoc@box}%
  \ifdim\@tempb\p@<\@tempa\p@\let\@tempa\@tempb\fi% select the smaller of both
  \ifdim\@tempa\p@<\p@\scalebox{\@tempa}{\usebox\pandoc@box}%
  \else\usebox{\pandoc@box}%
  \fi%
}
% Set default figure placement to htbp
\def\fps@figure{htbp}
\makeatother
\setlength{\emergencystretch}{3em} % prevent overfull lines
\providecommand{\tightlist}{%
  \setlength{\itemsep}{0pt}\setlength{\parskip}{0pt}}
\usepackage{bookmark}
\IfFileExists{xurl.sty}{\usepackage{xurl}}{} % add URL line breaks if available
\urlstyle{same}
\hypersetup{
  hidelinks,
  pdfcreator={LaTeX via pandoc}}

\author{}
\date{}

\begin{document}

\begin{longtable}[]{@{}
  >{\raggedright\arraybackslash}p{(\linewidth - 2\tabcolsep) * \real{0.3659}}
  >{\raggedright\arraybackslash}p{(\linewidth - 2\tabcolsep) * \real{0.6341}}@{}}
\toprule\noalign{}
\begin{minipage}[b]{\linewidth}\centering
\textbf{SỞ GIÁO DỤC VÀ ĐÀO TẠO}

\textbf{TỈNH PHÚ YÊN}

\textbf{ĐỀ SOẠN THẢO LẠI}

\emph{(Đề thi có 04 trang 04 bài)}
\end{minipage} & \begin{minipage}[b]{\linewidth}\centering
\textbf{KỲ THI CHỌN HỌC SINH GIỎI CẤP TỈNH}

\textbf{TRUNG HỌC PHỔ THÔNG, NĂM HỌC 2024 - 2025}

\textbf{Môn thi: TIN HỌC}

\textbf{Ngày thi:} \textbf{27/03/2025}

\textbf{Thời gian:} 180 phút \emph{(không kể thời gian phát đề)}

-\/-\/-\/-\/-\/-\/-\/-\/-\/-\/-\/-\/-\/-\/-\/-\/-\/-\/-\/-\/-
\end{minipage} \\
\midrule\noalign{}
\endhead
\bottomrule\noalign{}
\endlastfoot
\end{longtable}

\textbf{TỔNG} \textbf{QUAN BÀI THI}

\begin{longtable}[]{@{}
  >{\centering\arraybackslash}p{(\linewidth - 10\tabcolsep) * \real{0.0498}}
  >{\raggedright\arraybackslash}p{(\linewidth - 10\tabcolsep) * \real{0.3605}}
  >{\centering\arraybackslash}p{(\linewidth - 10\tabcolsep) * \real{0.1864}}
  >{\centering\arraybackslash}p{(\linewidth - 10\tabcolsep) * \real{0.1687}}
  >{\centering\arraybackslash}p{(\linewidth - 10\tabcolsep) * \real{0.1629}}
  >{\centering\arraybackslash}p{(\linewidth - 10\tabcolsep) * \real{0.0716}}@{}}
\toprule\noalign{}
\begin{minipage}[b]{\linewidth}\centering
\textbf{Bài}
\end{minipage} & \begin{minipage}[b]{\linewidth}\centering
\textbf{Tên bài}
\end{minipage} & \begin{minipage}[b]{\linewidth}\centering
\textbf{File chương trình}
\end{minipage} & \begin{minipage}[b]{\linewidth}\centering
\textbf{File dữ liệu vào}
\end{minipage} & \begin{minipage}[b]{\linewidth}\centering
\textbf{File dữ liệu ra}
\end{minipage} & \begin{minipage}[b]{\linewidth}\centering
\textbf{Điểm}
\end{minipage} \\
\midrule\noalign{}
\endhead
\bottomrule\noalign{}
\endlastfoot
\textbf{1} & \textbf{DÃY CON FIBONACCI} & \textbf{BAI1.*} &
\textbf{BAI1.INP} & \textbf{BAI1.OUT} & \textbf{5,00} \\
\textbf{2} & \textbf{KHOẢNG CÁCH HAMMING} & \textbf{BAI2.*} &
\textbf{BAI2.INP} & \textbf{BAI2.OUT} & \textbf{5,00} \\
\textbf{3} & \textbf{THỜI GIAN ĐOÀN XE QUA CẦU} & \textbf{BAI3.*} &
\textbf{BAI3.INP} & \textbf{BAI3.OUT} & \textbf{5,00} \\
\textbf{4} & \textbf{MÁY CHỦ THỨ BA} & \textbf{BAI4.*} &
\textbf{BAI4.INP} & \textbf{BAI4.OUT} & \textbf{5,00} \\
\end{longtable}

\emph{Học sinh phải đặt tên tệp tin chương trình , tên tệp tin dữ liệu
vào và tên tệp tin dữ liệu ra như phần tổng quan bài thi nêu trên. Dấu *
là CPP, PY, hoặc PAS, \ldots{} tương ứng đối với ngôn ngữ lập trình C++,
Python hoặc Pascal, ...}

\textbf{Bài 1:} \emph{(5,00 điểm)} \textbf{DÃY CON FIBONACCI}

Dãy số Fibonacci là một dãy số vô hạn được dịnh nghĩa theo công thức
sau:

A \(F_{0} = 0\)

A \(F_{1} = 1\)

A \(F_{n} = F_{n - 1} + F_{n - 2}\), với
\(n \geq 2,\ \ n\mathbb{\in N}\)

Điều này có nghĩa là, mỗi số trong dãy Fibonacci (bắt đầu từ số thứ ba
trở đi) bằng tổng của hai số liền kề trước đó. Ví dụ 10 số đầu tiên của
dãy Fibonacci là: \(0,\ 1,\ 1,\ 2,\ 3,\ 5,\ 8,\ 13,\ 21,\ 34\).

Nam viết lên bảng một số nguyên dương \(N\) và dãy số nguyên
\(a_{1},\ a_{2},\ldots,\ a_{N}\). Nam muốn biết trong dãy số trên có tồn
tại dãy con gồm các số liên tiếp có tổng là số Fibonacci hay không. Nếu
có thì độ dài của dãy con dài nhất là bao nhiêu?

\textbf{Yêu cầu:} Hãy tìm độ dài của dãy con dài nhất có tổng là số
Fibonacci.

\textbf{Dữ liệu vào:} Từ file văn bản \textbf{BAI1.INP} gồm:

\begin{itemize}
\item
  Dòng thứ nhất chứa số nguyên dương \(N\) (\(1 \leq N \leq 10^{4}\)).
\item
  Dòng thứ hai chứa \(N\) số nguyên \(a_{1},a_{2},\ldots,a_{N}\)
  (\(0 \leq a_{i} \leq 10^{9}\)).
\end{itemize}

\textbf{Kết quả:} Ghi ra file văn bản \textbf{BAI1.OUT} một số nguyên
duy nhất là độ dài của dãy con dài nhất tìm được.

\textbf{Ví dụ:}

\begin{longtable}[]{@{}
  >{\raggedright\arraybackslash}p{(\linewidth - 6\tabcolsep) * \real{0.3024}}
  >{\raggedright\arraybackslash}p{(\linewidth - 6\tabcolsep) * \real{0.3023}}
  >{\raggedright\arraybackslash}p{(\linewidth - 6\tabcolsep) * \real{0.3566}}
  >{\raggedright\arraybackslash}p{(\linewidth - 6\tabcolsep) * \real{0.0387}}@{}}
\toprule\noalign{}
\begin{minipage}[b]{\linewidth}\centering
\textbf{BAI1.INP}
\end{minipage} & \begin{minipage}[b]{\linewidth}\centering
\textbf{BAI1.OUT}
\end{minipage} & \begin{minipage}[b]{\linewidth}\centering
\textbf{Giải thích}
\end{minipage} & \begin{minipage}[b]{\linewidth}\centering
\end{minipage} \\
\midrule\noalign{}
\endhead
\bottomrule\noalign{}
\endlastfoot
7

3 1 2 4 1 5 4 & 5 & Dãy con \(1,\ 2,\ 4,\ 1,\ 5\) có tổng \(13\) là số
Fibonacci. & \\
\end{longtable}

\textbf{Bài 2:} \emph{(5,00 điểm)} \textbf{KHOẢNG CÁCH HAMMING}

Cho hai số tự nhiên \(a\), \(b\) được biểu diễn trong hệ thập phân. Ta
định nghĩa khoảng cách Hamming giữa hai số \(a\), \(b\) là
\(Hamming\ (a,\ b)\) bằng số lượng vị trí có chữ số khác nhau khi biểu
diễn \(a\), \(b\) trong hệ nhị phân có cùng số lượng chữ số. Lưu ý rằng
khi biểu diễn trong hệ nhị phân, số nào có số lượng chữ số ít hơn thì
thêm các chữ số \(0\) vào đầu để chúng có độ dài bằng nhau.

Ví dụ: với \(a = 13\), \(b = 25\) khi đó biểu diễn trong hệ nhị phân của
\(a\) là \(1101\) và của \(b\) là \(11001\), biểu diễn nhị phân của số
\(a\) ít hơn số \(b\) là \(1\) chữ số, vì vậy thêm \(1\) chữ số \(0\) ở
đầu để trở thành \(01101\). Lúc này ta có \(Hamming\ (13,25) = 2\) vì ở
vị trí \(1\) và vị trí \(3\) khác nhau trong biểu diễn nhị phân của
\(13\), \(25\).

\textbf{Yêu cầu:} Cho \(N\) số tự nhiên \(a_{1},\ a_{2},\ldots,a_{N}\)
được biểu diễn trong hệ thập phân. Hãy tính khoảng cách Hamming nhỏ nhất
giữa hai số bất kỳ trong dãy số trên.

\textbf{Dữ liệu vào:} Từ tập tin văn bản \textbf{BAI2.INP} gồm:

\begin{itemize}
\item
  Dòng thứ nhất chứa số nguyên dương \(N\)
  (\(2 \leq N \leq 2*10\hat{}4\)).
\item
  Dòng thứ hai chứa \(N\) số \(a_{1},\ a_{2},\ldots,a_{N}\)
  (\(0 \leq a_{i} \leq 10^{9}\)).
\end{itemize}

\textbf{Kết quả:} Ghi ra tập tin văn bản \textbf{BAI2.OUT} một số nguyên
duy nhất là khoảng cách Hamming nhỏ nhất giữa hai số bất kỳ trong dãy số
đã cho.

\textbf{Ví dụ:}

\begin{longtable}[]{@{}
  >{\raggedright\arraybackslash}p{(\linewidth - 6\tabcolsep) * \real{0.2607}}
  >{\raggedright\arraybackslash}p{(\linewidth - 6\tabcolsep) * \real{0.2609}}
  >{\raggedright\arraybackslash}p{(\linewidth - 6\tabcolsep) * \real{0.4450}}
  >{\raggedright\arraybackslash}p{(\linewidth - 6\tabcolsep) * \real{0.0334}}@{}}
\toprule\noalign{}
\begin{minipage}[b]{\linewidth}\centering
\textbf{BAI2.INP}
\end{minipage} & \begin{minipage}[b]{\linewidth}\centering
\textbf{BAI2.OUT}
\end{minipage} & \begin{minipage}[b]{\linewidth}\centering
\textbf{Giải thích}
\end{minipage} & \begin{minipage}[b]{\linewidth}\centering
\end{minipage} \\
\midrule\noalign{}
\endhead
\bottomrule\noalign{}
\endlastfoot
3

9 13 25 & 1 & \(Hamming(9,\ 13) = 1\)

\(Hamming(9,\ 25) = 1\)

\(Hamming(13,\ 25) = 2\)

Khoảng cách Hamming nhỏ nhất giữa hai số bất kỳ trong dãy số là \(1\).
& \\
\end{longtable}

\textbf{Ràng buộc:}

\begin{itemize}
\item
  Có \(60\%\) số test: \(N = 2\);
\item
  Có \(20\%\) số test: \(3 \leq N \leq 10^{3}\);
\item
  Có \(20\%\) số test: \(10^{3} < N \leq 2*10^{4}\).
\end{itemize}

\textbf{Bài 3:} \emph{(5,00 điểm)} \textbf{THỜI GIAN ĐOÀN XE QUA CẦU}

Người ta muốn xây dựng một khu du lịch trên một quần đảo xinh đẹp. Giữa
đào và đất liền chỉ có một chiếc cầu cho phép giao thông một chiều. Cầu
có trọng tải tối đa là \(P\) và chiều dài là \(L\). Một đoàn xe gồm
\(N\) chiếc được đánh số từ \(1\) đến \(N\) đang vận chuyển vật liệu từ
đất liền ra đảo, xe thứ \(i\) có tổng trọng lượng là \(w_{i}\) chạy với
vận tốc là \(v_{i}\). Để đảm bảo thời gian vận chuyển vật liệu và tải
trọng của cầu, cơ quan chức năng chỉ cho phép từng nhóm xe tuần tự qua
cầu. Nhóm xe sau chỉ được di chuyển sau khi toàn bộ xe của nhóm trước đã
qua cầu, các xe không được phép vượt nhau và tổng trọng lượng các xe
trong một nhóm không được vượt quá tải trọng của cầu. Thời gian qua cầu
của mỗi nhóm phụ thuộc vào xe có vận tốc thấp nhất trong nhóm.

\textbf{Yêu cầu:} Hãy tìm phương án chia \(N\) xe thành từng nhóm để
tổng thời gian di chuyển qua càu của cả đoàn xe là nhỏ nhất.

\textbf{Dữ liệu vào:} Từ file văn bản \textbf{BAI3.INP} gồm:

\begin{itemize}
\item
  Dòng thứ nhất ghi ba số nguyên \(N,P,L\)
  (\(1 \leq N \leq 1000,\ 1 \leq P \leq 100,\ 1 \leq L \leq 1000\)).
\item
  Dòng thứ \(i\) trong \(N\) dòng tiếp theo ghi hai số nguyên
  \(w_{i},\ v_{i}\) (\(1 \leq w_{i} \leq P,\ 1 \leq v_{i} \leq 100\)) là
  tổng trọng lượng và vận tốc của xe thứ \(i\).
\end{itemize}

\textbf{Kết quả:} Ghi ra file văn bản \textbf{BAI3.OUT} một số thực duy
nhất là thời gian nhỏ nhất tìm được (chính xác đến \(2\) chữ số thập
phân).

\textbf{Ví dụ:}

\begin{longtable}[]{@{}
  >{\raggedright\arraybackslash}p{(\linewidth - 6\tabcolsep) * \real{0.1812}}
  >{\raggedright\arraybackslash}p{(\linewidth - 6\tabcolsep) * \real{0.1884}}
  >{\raggedright\arraybackslash}p{(\linewidth - 6\tabcolsep) * \real{0.6057}}
  >{\raggedright\arraybackslash}p{(\linewidth - 6\tabcolsep) * \real{0.0246}}@{}}
\toprule\noalign{}
\begin{minipage}[b]{\linewidth}\centering
\textbf{BAI3.INP}
\end{minipage} & \begin{minipage}[b]{\linewidth}\centering
\textbf{BAI3.OUT}
\end{minipage} & \begin{minipage}[b]{\linewidth}\centering
\textbf{Giải thích}
\end{minipage} & \begin{minipage}[b]{\linewidth}\centering
\end{minipage} \\
\midrule\noalign{}
\endhead
\bottomrule\noalign{}
\endlastfoot
7 80 100

40 25

30 20

50 20

60 10

10 50

9 70

49 30 & 22.33 & - Nhóm \(1\) gồm xe \(1\); thời gian: \(100/25 = 4.00\)

- Nhóm \(2\) gồm xe \(2,\ 3\); thời gian: \(100/20 = 5.00\)

- Nhóm \(3\) gồm xe \(4,\ 5,\ 6\); thời gian: \(100/10 = 10.00\)

- Nhóm \(4\) gồm xe \(7\); thời gian: \(100/30\  = 3.33\)

Tổng thời gian qua cầu là:

\(4.00 + 5.00 + 10.00 + 3.33 = 22.33\) & \\
\end{longtable}

\textbf{Ràng buộc:}

\begin{itemize}
\item
  Có \(60\%\) số test: \(N \leq 10\);
\item
  Có \(30\%\) số test: \(N \leq 100\);
\item
  Có \(10\%\) số test: \(N \leq 1000\).
\end{itemize}

\textbf{Bài 4:} \emph{(5,00 điểm)} \textbf{MÁY CHỦ THỨ BA}

Hệ thống mạng máy tính của trung tâm Anpha có nhiệm vụ xử lí dữ liệu về
giao dịch thương mại điện tử cho các công ty khách hàng. Hệ thống này
được đặt ở nhiều toà nhà khác nhau trong thành phố. Hệ thống gồm có
\(N\) máy tính được đánh số tuần tự từ \(1\) đến \(N\). Giữa \(N\) máy
tính này là một mạng lưới gồm \(M\) đường truyền. Hai máy tính
\(x_{i}\), \(y_{i}\) bất kì được kết nối trực tiếp không quá một đường
truyền, có chi phí để truyền và xử lí dữ liệu là \(d_{i}\). Trong \(N\)
máy tính này thì máy tính \(1\), máy tính \(N\) là hai máy chủ và hệ
thống đường truyền đảm bảo luôn có ít nhất một cách truyền và xử lí dữ
liệu đi từ máy tính \(1\) đến máy tính \(N\).

Tuy nhiên, cả hai máy chủ của trung tâm Anpha này đều có dấu hiệu quá
tải về truyền và xử lí dữ liệu. Vì vậy, Ban giám đốc trung tâm quyết
định chọn ra thêm một máy tính nữa trong số các máy tính còn lại để nâng
cấp thành một máy chủ thứ ba. Máy tính này sẽ tạm ngưng hoạt động để
tiến hành nâng cấp, nhưng trong suốt thời gian nâng cấp yêu cầu phải đảm
bảo đường truyền và xử lí dữ liệu với chi phí ít nhất từ máy tính \(1\)
đến máy tính \(N\) được thông suốt, nếu không hệ thống mạng của Trung
tâm sẽ bị trì trệ ảnh hưởng đến hoạt động kinh doanh của các công ty
khách hàng.

\textbf{Yêu cầu:} Hãy giúp Ban giám đốc trung tâm Anpha xác định các máy
tính có thể được chọn làm máy chủ thứ ba sao cho máy tính được chọn thỏa
mãn các điều kiện ở trên.

\textbf{Dữ liệu vào:} Từ tập tin văn bản \textbf{BAI4.INP} gồm:

\begin{itemize}
\item
  Dòng đầu tiên ghi hai số nguyên dương \(N\), \(M\) là số lượng máy
  tính và số lượng đường truyền
  (\(2 \leq N \leq 3*10^{4},\ \ 1 \leq M \leq 10^{5}\));
\item
  Dòng thứ \(i\) trong số \(M\) dòng tiếp theo ghi ba số nguyên dương
  \(x_{i},y_{i},d_{i}\) với ý nghĩa chi phí truyền và xử lí dữ liệu giữa
  2 máy tính \(x_{i},y_{i}\) là \(d_{i}\)
  (\(1 \leq x_{i},y_{i} \leq N;\ 1 \leq d_{i} \leq 10^{3}\)).
\end{itemize}

\textbf{Kết quả:} Ghi ra tập tin văn bản \textbf{BAI4.OUT} gồm:

\begin{itemize}
\item
  Dòng đầu tiên ghi số nguyên \(S\) là số lượng các máy tính có thể được
  chọn làm máy chủ thứ ba;
\item
  \(S\) dòng tiếp theo, mỗi dòng ghi một số nguyên dương là số thứ tự
  của máy tính được chọn (in ra theo thứ tự tăng dần).
\end{itemize}

\begin{quote}
\textbf{Ví dụ:}
\end{quote}

\begin{longtable}[]{@{}
  >{\raggedright\arraybackslash}p{(\linewidth - 6\tabcolsep) * \real{0.1657}}
  >{\raggedright\arraybackslash}p{(\linewidth - 6\tabcolsep) * \real{0.1722}}
  >{\raggedright\arraybackslash}p{(\linewidth - 6\tabcolsep) * \real{0.6396}}
  >{\raggedright\arraybackslash}p{(\linewidth - 6\tabcolsep) * \real{0.0226}}@{}}
\toprule\noalign{}
\begin{minipage}[b]{\linewidth}\centering
\textbf{BAI4.INP}
\end{minipage} & \begin{minipage}[b]{\linewidth}\centering
\textbf{BAI4.OUT}
\end{minipage} & \begin{minipage}[b]{\linewidth}\centering
\textbf{Hình vẽ}
\end{minipage} & \begin{minipage}[b]{\linewidth}\centering
\end{minipage} \\
\midrule\noalign{}
\endhead
\bottomrule\noalign{}
\endlastfoot
8 10

1 5 3

2 3 2

5 3 1

5 6 1

5 2 1

6 2 1

1 2 3

6 7 1

7 8 1

2 4 1 & 4

2

3

4

5 & \includegraphics[width=4.28819in,height=2.43182in]{media/image1.png}
& \\
\end{longtable}

\begin{quote}
\textbf{Ràng buộc:}
\end{quote}

\begin{itemize}
\item
  Có \(80\%\) số test: \(2 \leq N \leq 3*10^{3}\);
\item
  Có \(20\%\) số test: \(3*10^{3} < N \leq 3*10^{4}\);
\end{itemize}

\emph{\textbf{Lưu ý:} Tất cả các giá trị số trên cùng một dòng của file
dữ liệu vào, file dữ liệu ra ghi cách nhau một dấu cách.}

\textbf{-\/-\/-\/-\/-\/-\/-\/-\/-\/- HẾT -\/-\/-\/-\/-\/-\/-\/-\/-\/-}

\emph{\textbf{Chúc bạn làm bài tốt!!:)}}

\emph{\textbf{(Đề được soạn thảo lại bởi ĐKA - học sinh trường THPT Lê
Lợi!)}}

\end{document}
